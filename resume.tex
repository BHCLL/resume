%!TEX program = xelatex
\documentclass[letterpaper,11pt]{article}

% [a4paper]
\usepackage[centering,paperwidth=210mm,paperheight=297mm,body={180mm,257mm}]{geometry}

\usepackage{
  hyperref,
  color,
  latexsym,
  verbatim,
  url,
  ulem,
  xeCJK, % replace with CJK for sharelatex.com
  multirow,
  enumitem,
  calc % fix hbox too wide for heading, enable caculate
}

% \usepackage{latexsym,verbatim,url,CJKutf8}
\pagestyle{empty}
\urlstyle{same}

\hypersetup{
    colorlinks,%
    citecolor=black,%
    filecolor=black,%
    linkcolor=black,%
    urlcolor=black     % can put red here to better visualize the links
}
\definecolor{mygrey}{gray}{.9}
\definecolor{mygreylink}{gray}{.30}
\definecolor{labelgrey}{gray}{.50}

\raggedbottom
\raggedright
\setlength{\tabcolsep}{0in}
\setlength{\baselineskip}{15.6pt}

%-----------------------------------------------------------
%Custom commands
\newcommand{\resitem}[1]{\item #1 \vspace{-2pt}}
\newcommand{\resheading}[1]{{\large \colorbox{mygrey}{\begin{minipage}{\textwidth-2\fboxsep}{\textbf{#1 \vphantom{p\^{E}}}}\end{minipage}}}}
\newcommand{\ressubheading}[4]{
\begin{tabular*}{\textwidth-5mm}{l@{\extracolsep{\fill}}r}
    \textbf{#1} & #2 \\
    \textit{#3} & #4 \\
\end{tabular*}\vspace{-6pt}}

% \newcommand{\ressubsingleheading}[3]{
% \begin{tabular*}{\textwidth-5mm}{@{\extracolsep{\fill}}lr}
%     \multirow{2}{*}{\textbf{#1}} & #2 \\
%     & #3 \\
% \end{tabular*}\vspace{-6pt}}

\newcommand{\ressubsingleline}[3]{
\begin{tabular*}{\textwidth-5mm}{ll@{\extracolsep{\fill}}r}
    \textbf{#1} & \quad\textit{#2} & #3 \\
\end{tabular*}}


\begin{document}
% \begin{CJK*}{UTF8}{gbsn} % uncomment  for sharelatex.com

\newcommand{\myheader}{
\begin{tabular*}{7.0in}{l@{\extracolsep{\fill}}r}
  \textbf{\href{http://herechen.github.io}{\LARGE 陈磊}} & 15196620528$\,${\color{labelgrey}(电话)} \\
  四川省成都市二环路北一段111号$\,${\color{labelgrey}(地址)} & \href{mailto:chenlei.here@qq.com}{chenlei.here@qq.com}$\,${\color{labelgrey}(邮件)} \\
  610031$\,${\color{labelgrey}(邮编)} & \href{http://herechen.github.io}{http://herechen.github.io}$\,${\color{labelgrey}(主页)} \\
  \end{tabular*}\\\vspace{0.1in}}

\myheader

\resheading{教育经历}
  \begin{itemize}[leftmargin=*]
    \item
      \ressubheading{西南交通大学}{硕士学位}{信息科学与技术学院$\cdot$计算机技术}{2013.09 -- 2016.06}
      % {\footnotesize
      % \begin{itemize}
      %   % \item{专业课程:  现代服务模式, 基于 XML 的产品数据交互技术, 3G 智能手机 web 应用开发技术, 嵌入式系统, 复杂系统建模与应用, 供应链管理系统设计与开发实践, 价值网与云服务平台技术, 基于 web 的协同平台设计}
      %   \item 毕业论文: 《汽车售后服务配件多级库存调拨系统设计与实现》
      % \end{itemize}
      % }
    \item
      \ressubheading{西南交通大学}{学士学位}{数学学院$\cdot$信息与计算科学}{2009.09 -- 2013.06}
      % {\footnotesize
      % \begin{itemize}
      %   % \item{专业课程: 软件工程实践, 数据库与程序设计实践, 编程实践, VB 程序设计, 数据库原理与设计, 数据结构与算法, 程序设计与算法语言, 大学计算机基础,计算智能, 数据分析, 数学分析, 高等代数, 泛函分析, 复变函数, 数学建模, 概率统计, 近世代数, 实变函数, 数值分析, 数理方程, 应用模糊数学, 运筹学基础, 最优化算法, 控制论基础, 信息论基础, 最优控制理论, 常微分方程, 微分方程数值解, 几何学}
      %   \item 毕业论文: 《自适应重启的加速梯度法的实现及应用》
      % \end{itemize}
      % }
  \end{itemize}

\resheading{项目经历}
  \begin{itemize}[leftmargin=*]
    \item
      \ressubheading{制造业产品设计服务与产业链协同技术研发与应用示范$\,$(国家支撑计划)}{前端开发}{汽车及零部件中小企业产业链协同平台}{2015.04 -- 2015.08}
      {\footnotesize
      \begin{itemize}
        \item 开发环境: Windows 7,Visual Studio 2010,SQL Server 2008,AnkhSVN
        \item 项目描述: 该项目是西南交通大学承担的课题。产业链协同系统面向汽车产业,统共分为整车厂专区、经销商专区、配件交易专区以及两个后台管理系统几个模块。项目开发采用 ASP.NET WebForms,部分页面应用了响应式设计。
        \item 主要任务: 搭建项目模板页,用 Phtoshop 切分设计图,完成页面样式和 JavaScript 编写,通过 IE 测试虚拟机完成页面样式的兼容性测试。
        \item 项目成果:已结项上线$\,$(地址: www.scmie.com.cn -- 汽车及零部件产业链中小企业应用平台)
      \end{itemize}
      }
    \item
      \ressubheading{支撑区域和地方支柱产业的制造业信息化综合应用示范$\,$(国家支撑计划)}{前端开发}{四川省制造业信息化企业服务平台}{2015.04 -- 2015.08}
      {\footnotesize
      \begin{itemize}
        \item 开发环境: Windows 7,eclipse,MySQL,TortoiseSVN
        \item 项目描述: 该项目是西南交通大学承担的课题。服务平台为四川省重点制造业企业提供制造业行业相关新闻动态、示范企业信息化建设参考、行业软件的试用的试用服务以及相关软件的下载服务。项目使用 Java 语言开发,并采用了 SpringMVC 和 JPA 开源技术框架。
        \item 主要任务: 页面布局设计,通过 Phtoshop 切分设计图,页面样式和 JavaScript 编写,页面样式兼容性测试。
        \item 项目成果: 已结项上线$\,$(地址: www.scmie.com.cn)
      \end{itemize}
      }
    \item
    \ressubsingleheading{天圆地方在线点餐服务网站}{前端开发}{2014.09 -- 2014.12}
    {\footnotesize
    \begin{itemize}
      \item 开发环境: Windows 7, eclipse, MySQL, TortoiseSVN
      \item 项目描述: 该项目是为贵州黎平县一餐饮企业设计开发、支持多店铺展示,客户可以在线订餐,由相应的餐馆负责送餐。该项目使用 Java 语言,采用 Struts2、Spring、Hibernate 框架开发。
      \item 主要任务: 页面布局设计, 页面样式和 JavaScript 编写。
      \item 项目成果: 已结项
    \end{itemize}
    }
    \item
    \ressubsingleheading{面向盾构机的零部件采购到货协同管理系统}{全栈开发}{2014.05 -- 2014.08}
    {\footnotesize
    \begin{itemize}
      \item 开发环境: Windows Server 2003,Visual Studio 2010,SQL Server 2008
      \item 项目描述: 该项目原型是成都国龙信息工程有限责任公司开发的中国铁建重工零部件采购平台。 平台分为售后和采购两部分。实习期间,根据采购部分的采购到货流程,采用 ASP.NET WebForms 开发了面向盾构机的零部件采购到货协同管理系统。
      \item 主要任务: 依据采购业务独立出完整的采购到货业务流程,并完成系统的设计和实现。
    \end{itemize}
    }
    \item \ressubsingleheading{手持医疗终端}{嵌入式开发}{2012.07}
    {\footnotesize
    \begin{itemize}
      \item 开发环境: Windows XP, VMvare, Red Hat 9, Qt
      \item 项目描述: 这是在东软的暑期实训项目。手持医疗终端项目目标,可以在移动端实现患者血糖血压的收集,并提供医院的相关信息,在 Web 端登录能够查看患者信息。实际开发通过在 Window 安装虚拟机,并安装 Linux,通过 crosstool 实现交叉编译,在 2410-S 上测试。
      \item 主要任务: 嵌入式交叉编译环境搭建,用 C++ 编写界面,并集成单片机数据模拟程序。
    \end{itemize}
    }
  \end{itemize}

\resheading{实习经历}
  \begin{itemize}[leftmargin=*]
    \item
      \ressubsingleline{成都国龙信息工程有限责任公司}{研发部-程序员}{2014.05 -- 2014.11}
  \end{itemize}

\resheading{校园实践}
  \begin{itemize}[leftmargin=*]
    \item
      \ressubsingleline{雷锋公司协会}{会员,副部长}{2009.09 -- 2012.06}
    \item
      \ressubsingleline{数学学院青年志愿者协会}{会员}{2009.09 -- 2010.06}
  \end{itemize}

\resheading{资格证书}
  \begin{itemize}[leftmargin=*]
    \item
      \ressubsingleline{全国计算机等级}{二级C,二级C++,三级数据库技术}{2011.03 -- 2012.03}
    \item
      \ressubsingleline{全国大学英语}{四级 (549),六级 (464)}{2010.06 -- 2011.12}
  \end{itemize}

\resheading{获奖情况}
  \begin{itemize}[leftmargin=*]
    \item \ressubheading{高教社杯全国大学生数学建模竞赛$\,$(本科组)}{四川赛区$\,$一等奖}{《葡萄酒的评价》}{2012.09}
    \item \ressubheading{西南交通大学第六期大学生科研训练计划项目}{校级优秀项目}{《公共养老保险风险建模与应用分析》}{2012.05}
    \item \ressubheading{高教社杯全国大学生数学建模竞赛$\,$(本科组)}{全国$\,$二等奖}{《交巡警服务平台的设置与调度》}{2011.09}
    \item \ressubheading{“中国电机工程学会杯” 全国大学生电工数学建模竞赛}{全国$\,$三等奖}{《风功率预测问题》}{2011.12}
    \item \ressubheading{第八届苏北数学建模联赛}{全国$\,$二等奖}{《旅游线路的优化设计》}{2011.05}
    \item \ressubheading{西南交通大学数学建模竞赛}{二等奖}{《成都市三环线——绕城高速西北区域公交路线的设计》}{2011.05}
    \item \ressubsingleline{数学学院优秀学生干部}{}{2011.11}
    \item \ressubsingleline{精神文明建设积极分子}{}{2010.11}
    \item \ressubsingleline{西南交通大学综合奖学金}{三次二等,两次三等}{2009.09 -- 2013.06}
  \end{itemize}

\resheading{专业技能}
  \begin{itemize}[leftmargin=*]
    \item \textbf{英语}: 能够流畅的阅读科技博文,并熟练的用英文检索开发中遇到的问题
    \item \textbf{语言}: C/C++,C\#,HTML,CSS,JavaScript,\href{http://cn.mathworks.com/products/matlab/}{MATLAB},\href{http://www.latex-project.org/}{\LaTeX}
    \item \textbf{系统}: Windows,Ubuntu
    \item \textbf{软件/工具}: MS Visual Studio,Chrome$\,$(开发工具),Firefox$\,$(开发工具),Photoshop,Sublime Text,Atom,Git,SVN
    \item \textbf{项目/框架}: \href{http://www.bootcss.com/}{Bootstrap},\href{https://jquery.com/}{jQuery},\href{http://jeasyui.com/}{jQuery EasyUI},less,sass,\href{http://gulpjs.com/}{gulp}
    % ,\href{https://angularjs.org/}{AngularJS},\href{http://gulpjs.com/}{gulp},sass/compass
    \item \textbf{网站开发}: ASP.NET Web Forms,ASP.NET MVC
  \end{itemize}

% \resheading{其他}
%   \begin{description}
%     \item[兴趣:] 哲学(马克思, 尼采), 数学(方程数值解, 最优化算法), 在浏览网页时按 F12, MATLAB 代码优化, \LaTeX 排版
%     \item[学习新事物的方法:] 了解 $\rightarrow$ 知道 $\rightarrow$ 理解
%     \item[价值观和方法论:] 朴素 $\cdot$ 生命力 $\cdot$ 自我批判
%   \end{description}
% \end{CJK*}

\end{document}
