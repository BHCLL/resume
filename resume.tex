%!TEX program = xelatex
\documentclass[letterpaper,11pt]{article}

% [a4paper]
\usepackage[centering,paperwidth=210mm,paperheight=297mm,body={180mm,257mm}]{geometry}

\usepackage{
  hyperref,
  color,
  latexsym,
  verbatim,
  url,
  ulem,
  xeCJK, % replace with CJK for sharelatex.com
  multirow,
  enumitem,
  calc % fix hbox too wide for heading, enable caculate
}

% \usepackage{latexsym,verbatim,url,CJKutf8}
\pagestyle{empty}
\urlstyle{same}

\hypersetup{
    colorlinks,%
    citecolor=black,%
    filecolor=black,%
    linkcolor=black,%
    urlcolor=black     % can put red here to better visualize the links
}
\definecolor{mygrey}{gray}{.9}
\definecolor{mygreylink}{gray}{.30}
\definecolor{labelgrey}{gray}{.50}

\raggedbottom
\raggedright
\setlength{\tabcolsep}{0in}
\setlength{\baselineskip}{15.6pt}

%-----------------------------------------------------------
%Custom commands
\newcommand{\resitem}[1]{\item #1 \vspace{-2pt}}
\newcommand{\resheading}[1]{{\large \colorbox{mygrey}{\begin{minipage}{\textwidth-2\fboxsep}{\textbf{#1 \vphantom{p\^{E}}}}\end{minipage}}}}
\newcommand{\ressubheading}[4]{
\begin{tabular*}{\textwidth-5mm}{l@{\extracolsep{\fill}}r}
    \textbf{#1} & #2 \\
    \textit{#3} & #4 \\
\end{tabular*}\vspace{-6pt}}

% \newcommand{\ressubsingleheading}[3]{
% \begin{tabular*}{\textwidth-5mm}{@{\extracolsep{\fill}}lr}
%     \multirow{2}{*}{\textbf{#1}} & #2 \\
%     & #3 \\
% \end{tabular*}\vspace{-6pt}}

\newcommand{\ressubsingleline}[3]{
\begin{tabular*}{\textwidth-5mm}{ll@{\extracolsep{\fill}}r}
    \textbf{#1} & \quad\textit{#2} & #3 \\
\end{tabular*}}


\begin{document}
% \begin{CJK*}{UTF8}{gbsn}

\newcommand{\myheader}{
\begin{tabular*}{7.0in}{l@{\extracolsep{\fill}}r}
  \textbf{\href{http://herechen.github.io}{\LARGE 陈磊}} & 15196620528$\,${\color{labelgrey}(电话)} \\
  四川省成都市二环路北一段111号$\,${\color{labelgrey}(地址)} & \href{mailto:chenlei.here@qq.com}{chenlei.here@qq.com}$\,${\color{labelgrey}(邮件)} \\
  610031$\,${\color{labelgrey}(邮编)} & \href{http://herechen.github.io}{http://herechen.github.io}$\,${\color{labelgrey}(主页)} \\
  \end{tabular*}\\\vspace{0.1in}}

\myheader

\resheading{教育}
  \begin{itemize}
    \item
      \ressubheading{\href{http://www.swjtu.edu.cn}{西南交通大学}}{\sout{硕士学位}}{信息科学与技术学院・计算机技术}{2013.09 -- 当前}
      % {\footnotesize
      % \begin{itemize}
      %   % \resitem{专业课程:  现代服务模式, 基于 XML 的产品数据交互技术, 3G 智能手机 web 应用开发技术, 嵌入式系统, 复杂系统建模与应用, 供应链管理系统设计与开发实践, 价值网与云服务平台技术, 基于 web 的协同平台设计}
      %   \item 毕业论文:库存调拨
      % \end{itemize}
      % }
    \item
      \ressubheading{\href{http://www.swjtu.edu.cn}{西南交通大学}}{学士学位}{数学学院・信息与计算科学}{2009.09 -- 2013.06}
      {\footnotesize
      \begin{itemize}
        % \resitem{专业课程: 计算智能, 最优化算法, 信息论基础, 数据分析, 最优控制理论, 数学分析, 高等代数, 泛函分析, 几何学, 数学建模, 概率统计, 近世代数, 实变函数, 数值分析, 数理方程, 应用模糊数学, 运筹学基础, 控制论基础, 复变函数, 常微分方程, 微分方程数值解 }
        \item 毕业论文:自适应重启的加速梯度法
      \end{itemize}
      }
  \end{itemize}

\resheading{项目}
  \begin{itemize}
    \item
      \ressubheading{制造业产品设计服务与产业链协同技术研发$\,$(国家支撑计划)}{前端开发}{汽车及零部件中小企业产业链协同平台}{2015.04--2015.07}
      {\footnotesize
      \begin{itemize}
        \item{工作内容:}
      \end{itemize}
      }
    \item
      \ressubheading{支撑区域和地方支柱产业的制造业信息化综合应用示范$\,$(国家支撑计划)}{前端开发}{四川省制造业信息化企业服务平台}{2015.04--2015.07}
      {\footnotesize
      \begin{itemize}
        \item 平台为四川省重点制造业企业提供制造业行业相关新闻动态、示范企业信息化建设参考、行业软件的试用的试用服务以及相关软件的下载服务。
      \end{itemize}
      }
    \item
    \ressubheading{天圆地方在线点餐服务网站}{前端开发}{xxx}{2014.08--2014.11}
    {\footnotesize
    \begin{itemize}
      \item 该项目是为贵州黎平县一餐饮企业设计开发、支持多店铺展示,客户可以在线订餐,由相应的餐馆负责送餐。
    \end{itemize}
    }
  \end{itemize}

\resheading{获奖}
  \begin{itemize}
    \item \ressubheading{高教社杯全国大学生数学建模竞赛$\,$(本科组)}{四川省$\,$一等奖}{葡萄酒的评价}{2012.09}
    \item \ressubheading{高教社杯全国大学生数学建模竞赛$\,$(本科组)}{全国$\,$二等奖}{交巡警服务平台的设置与调度}{2011.09}
    % \item \ressubheading{“中国电机工程学会杯”全国大学生电工数学建模竞赛}{三等奖}{XXX}{2011.12}
    % \item \ressubheading{第八届苏北数学建模联赛}{二等奖}{XXX}{2011.05}
    % \item \ressubheading{西南交通大学数学建模竞赛}{二等奖}{XXX}{2011.05}
    % \item \ressubheading{西南交通大学数学建模首届“新秀杯”}{三等奖}{XXX}{2010.12}
    \item \ressubheading{西南交通大学综合奖学金}{}{三次二等、两次三等}{2009.09--2013.06}
  \end{itemize}

\resheading{证书}
  \begin{itemize}
    \item
      \ressubheading{全国计算机等级}{}{二级C、二级C++、三级数据库技术}{2011.03--2012.03}
    \item
      \ressubheading{全国大学英语}{}{四级、六级}{2010.06--2011.12}
  \end{itemize}
    % \begin{description}
    %   \item[\textit{2011.03--2012.03}] 全国计算机等级$\;$二级C、二级C++、三级数据库技术
    %   \item[\textit{2010.06--2011.12}] 全国大学英语$\;$四级、六级
    % \end{description}

\resheading{技能}
  \begin{description}
    \item[语言:] HTML, CSS, JavaScript, C/C++, C\#, \href{http://cn.mathworks.com/products/matlab/}{MATLAB}, \href{http://www.latex-project.org/}{\LaTeX}
    \item[系统:] Windows, Ubuntu
    \item[软件:] MS Visual Studio, Chrome$\,$(开发工具), Firefox$\,$(开发工具), Sublime Text, Atom
    \item[项目/框架:] \href{http://www.bootcss.com/}{Bootstrap}, \href{https://jquery.com/}{jQuery}, \href{https://angularjs.org/}{AngularJS}, \href{http://gulpjs.com/}{gulp}, sass/compass
    \item[网站开发:] ASP.NET Web Forms, ASP.NET MVC
  \end{description}

% \end{CJK*}

\end{document}
